% LaTeX file for resume
% This file uses the resume document class (res.cls)

\documentclass{res}
\usepackage{helvetica} % uses helvetica postscript font (download helvetica.sty)
\usepackage{newcent}   % uses new century schoolbook postscript font
\newsectionwidth{0pt}  % So the text is not indented under section headings
\usepackage{fancyhdr}  % use this package to get a 2 line header
\renewcommand{\headrulewidth}{0pt} % suppress line drawn by default by fancyhdr
\setlength{\headheight}{24pt} % allow room for 2-line header
\setlength{\headsep}{24pt}  % space between header and text
\setlength{\headheight}{24pt} % allow room for 2-line header
\pagestyle{fancy}     % set pagestyle for document
\rhead{ {\it Neil Huang}\\{\it p. \thepage} } % put text in header (right side)
\cfoot{}                                     % the foot is empty
\topmargin=-1.3in % start text higher on the page

\begin{document}
\thispagestyle{empty} % this page has no header
\name{Yao-Chih Huang(Neil Huang)\\[2pt]}% the \\[12pt] adds a blank line after name

\address{Email: kkoala0864@gmail.com}

\begin{resume}

\vspace{0.01in}
\section{\centerline{WORK EXPERIENCE}}
%\rule{\textwidth}{1mm}
\noindent\rule[0.25\baselineskip]{\textwidth}{1pt}
\vspace{6pt}
{\bf Synology}  \hfill Taipei, Taiwan\\
{\sl Software Engineer / Senior Software Engineer       \hfill Aug 2014 - Dec 2018 / Jan 2019 - Present}

\begin{itemize} \itemsep +4pt
   \item {\bf Cross-Platform Toolchain Unify}\\
   Designed and implemented a cross-platform toolchain building flow for applying security patches to protect Synology products from security attacks.
   Unified toolchains for ARM and PowerPC platforms, addressing and resolving related issues.

   \item {\bf ARM Based SOC Bring Up}\\
   Bring up Marvell and STMicroelectronics SoCs, including kernel porting, hardware function implementation, performance tuning, and bug fixes.
   Successfully delivered Synology NAS product DS115, RS815, and DS216Play.

   \item {\bf Cloud Directory Serivce}\\
   Developed a directory service on the Synology C2 cloud platform from scratch, including surveying and designing product specifications. Constructed the service backend on Kubernetes and designed user-friendly tools for service accessibility. Designed a user agent to download LDAP server data from the cloud to end users, providing local authentication services. Implemented the SRP (Secure Remote Password) protocol as the authentication algorithm to ensure no user passwords are stored at the service backend.

   \item {\bf VPN Service}\\
   Maintaining Synology VPN service application, including troubleshooting and performance tuning make application 70\% speed up.

   \item {\bf Backup Service}\\
   Developed a backup service application for Synology NAS products, enabling comprehensive backup solutions for user computers and servers. Implemented disaster recovery features to ensure data integrity and restoration capabilities from the Synology server.

\end{itemize}

{\bf Marvell Semiconductor}  \hfill Hsinchu, Taiwan\\ 
{\sl Software Engineer \hfill     Oct 2011 - Jun 2014}

\begin{itemize} \itemsep +4pt
  \item {\bf ARMv8 Architecture Simulator}\\
  Implementing ARMv8 Instruction-Level accurate simulator from scratch.
  Designing test framework that generate test pattern and compare with golden model (ARM Fast Model) at runtime to verify simulator accuracy.

  \item {\bf ActionScript Virtual Machine}\\
  Porting Adobe AVM(ActionScript Virtual Machine) to ARM Target.
  Resolving issue to make AVM can execute ActionScript byte code on ARM based development board and passing regression test.

  \item {\bf Marvell Compiler Toolchain Testing On Android Platform}\\
  Through building Android system image by every Marvell toolchain release version to test Marvell compiler toolchain.
  Troubleshooting to make system image can execute on ARM based development board successfully.

\end{itemize} \vspace{-6pt}

\vspace{0.1in}
\section{\centerline{EDUCATION}}
\noindent\rule[0.25\baselineskip]{\textwidth}{1pt}
\vspace{3pt}
{\sl M.S., Computer Science \hspace{0.2in} \hfill Sep 2008 - Sep 2010} \\
{National Chiao Tung University} \hspace{0.2in} \hfill Hsinchu, Taiwan \\
Thesis: File-Based Sharing For Dynamically Compiled Code on Dalvik Virtual Machine

\end{resume}
\end{document}

