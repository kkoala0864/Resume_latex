% LaTeX file for resume
% This file uses the resume document class (res.cls)

\documentclass{res}
\usepackage{helvetica} % uses helvetica postscript font (download helvetica.sty)
\usepackage{newcent}   % uses new century schoolbook postscript font
\newsectionwidth{0pt}  % So the text is not indented under section headings
\usepackage{fancyhdr}  % use this package to get a 2 line header
\renewcommand{\headrulewidth}{0pt} % suppress line drawn by default by fancyhdr
\setlength{\headheight}{24pt} % allow room for 2-line header
\setlength{\headsep}{24pt}  % space between header and text
\setlength{\headheight}{24pt} % allow room for 2-line header
\pagestyle{fancy}     % set pagestyle for document
\rhead{ {\it Neil Huang}\\{\it p. \thepage} } % put text in header (right side)
\cfoot{}                                     % the foot is empty
\topmargin=-0.5in % start text higher on the page

\begin{document}
\thispagestyle{empty} % this page has no header
\name{Yao-Chih Huang(Neil Huang)\\[12pt]}% the \\[12pt] adds a blank line after name

\address{Email: kkoala0864@gmail.com}

\begin{resume}

\vspace{0.2in}
\section{\centerline{WORK EXPERIENCE}}
%\rule{\textwidth}{1mm}
\noindent\rule[0.25\baselineskip]{\textwidth}{1pt}
\vspace{8pt}
{\bf Synology}  \hfill Taipei, Taiwan\\
{\sl Software Engineer / Senior Software Engineer       \hfill Oct 2014 - Nov 2018 / Jan 2019 - Present}

\begin{itemize} \itemsep +5pt
   \item {\bf Cloud Directory Serivce}\\
   Developing directory service on Synology cloud platform (C2) from scratch, including survey and design product specification, construct service backend on Kubernetes, and design tool to make user easy to use service.
   Designing user agent which downloads LDAP server data from cloud to end user and provides local authentication service.
   Implementing SRP (Secure Remote Password) protocol as authentication algorithm to avoid storing user password at service backend.

   \item {\bf VPN Service}\\
   Maintaining Synology VPN service application, including troubleshooting and performance tuning make application 70\% speed up.

   \item {\bf Cross-Platform Toolchain Unify}\\
   Designing cross-platform toolchain building flow to apply security patch, avoid Synology product to suffer security attack. Also unify ARM and PowerPC platform toolchain and related issue fix.

   \item {\bf ARM Based SOC Bring Up}\\
   Marvell and STMicroelectronics SOC bring up, including kernel porting, hardware function implementation, performance tuning and relative bug fix. Deliver Synology DS115, RS815, DS216Play Model.
\end{itemize}

{\bf Marvell Semiconductor}  \hfill Hsinchu, Taiwan\\ 
{\sl Software Engineer \hfill     Oct 2011 - Jun 2014}

\begin{itemize} \itemsep +5pt
  \item {\bf ARMv8 Architecture Simulator}\\
  Implementing ARMv8 Instruction-Level accurate simulator from scratch.
  Designing test framework that generate test pattern and compare with golden model (ARM Fast Model) at runtime to verify simulator accuracy.

  \item {\bf ActionScript Virtual Machine}\\
  Porting Adobe AVM(ActionScript Virtual Machine) to ARM Target.
  Resolving issue to make AVM can execute ActionScript byte code on ARM based development board and passing regression test.
\end{itemize} \vspace{-6pt}

\vspace{0.2in}
\section{\centerline{EDUCATION}}
\noindent\rule[0.25\baselineskip]{\textwidth}{1pt}
\vspace{8pt}
{\sl M.S., Computer Science \hspace{0.2in} \hfill Sep 2008 - Sep 2010} \\
{National Chiao Tung University} \hspace{0.2in} \hfill Hsinchu, Taiwan \\
Thesis: File-Based Sharing For Dynamically Compiled Code on Dalvik Virtual Machine

{\sl B.S., Computer Science \hspace{0.2in} \hfill Sep 2003 - Jun 2008} \\
{Fu-Jen Catholic University} \hspace{0.2in} \hfill Taipei, Taiwan \\
\end{resume}
\end{document}

